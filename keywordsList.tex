\documentclass{article}
\usepackage{enumitem,amssymb}
\newlist{todolist}{itemize}{2}
\setlist[todolist]{label=$\square$}
\begin{document}
	
{\huge \textbf{6th Grade Tech Class}}

\vspace{1cm}

{\huge \textbf{Vocabulary Sheet}}

\begin{itemize}
  \item \textbf{Git} - A Version Control System for managing text files
  \item \textbf{GitHub} - A website where git repositories are shared
  \item \textbf{Repository} - A collection of files and their save-points
  \item \textbf{Github Project} - A collection of repositories 
  \item \textbf{Repository Fork} - A copy of a repository that you can edit
  \item \textbf{Pull Request} - A request to bring changes from a forked repository into the origin
  \item \textbf{Github Issue} - A task in the TODO list within Github
  \item \textbf{Linux }- An operating system for most computers in the world
  \item \textbf{Ad blocker} - A necessary safety filter for web browsers
  \item \textbf{Filesystem }- All of the files and folders on a specific computer
  \item \textbf{Cheese (Program)} - A program for taking pictures with the Webcam
  \item \textbf{Krita (Program) }- A program for cropping and resizing images
  \item \textbf{IDE} - Integrated Development Environment used for editing and uploading programs
  \item \textbf{Microcontroller} - A tiny computer that can run just one program
  \item \textbf{Arduino} (Program) - An IDE used for programming microcontrollers
  \item \textbf{Function} (in programming context) - A set of code that can be called to run
  \item \textbf{Variable} (in programming context) - A "bucket" that holds some data for later use
  \item \textbf{Integer} - A whole number
  \item \textbf{Digital} - Values that can be either 1 or 0
  \item \textbf{Analog} - Values that can be a range of values greater than just 1 and 0
  \item \textbf{Servo} - A motor with a built in controller that can be given position or speed commands
  \item \textbf{Serial Monitor} - A tab in Arduino IDE to see print statements
  \item \textbf{IMU} - Inertial Measurement Unit - a sensor for measuring motion and orientation
  \item \textbf{Semi-colon} - ; Put at the end of a line that executes (the key to the right of L)
  \item \textbf{Open Parentheses} - ( opening of the functions parameter section (SHIFT + 9)
  \item \textbf{Close Parentheses} - ) closing the functions parameter section (SHIFT + 0)
  \item \textbf{Open Curly Brace} - \{ begin the body of a function (SHIFT + the key to the right of P)
  \item \textbf{Close Curly Brace} - \} close the body of a function (SHIFT + the key to the right of \{ )
  \item \textbf{Open Bracket} - [\ begin a list (the key to the right of P)
  \item \textbf{Close Bracket} - ]\ end a list ( the key to the right of [\ )
  
\end{itemize}

\newpage
\section{6th Grade Tech Class One TODO:}
\begin{todolist}
	\item Create Github Account
\end{todolist}


\section{6th Grade Tech Class Two TODO:}
\begin{todolist}
	\item Fork LabCodeRepoSetup
	\item Add Github Username to art6th.json
	\item Create Pull Request
\end{todolist}


\section{6th Grade Tech Class Three TODO:}
\begin{todolist}
	\item Login to Github Desktop
	\item Create an issue for update readme
	\item Do the changes to the readme
	\item Commit changes with issue in message
	\item Push in Github Desktop
	\item Observe issue closed by commit 
\end{todolist}


\section{6th Grade Tech Class Four TODO:}
\begin{todolist}
	\item Make  Github issue to add a photo
	\item Use Cheese to take a photo
	\item Locate picture on the filesystem
	\item Open With Krita
	\item use Rectangular selection tool to select crop area
	\item use Image menu to "Trim to selection"
	\item use Image menu to "Scale image to new size" (600 width) 
	\item Save image in Girhub repository working directory
	\item Add image to readme using markdown syntax
	\item commit changes with issue in the message
	\item push changes
\end{todolist}

\section{6th Grade Tech Class Five TODO:}
\begin{todolist}
	\item Login to Type to Learn
	\item Aim for 95 percent accuracy
	\item Aim for at least 20 WPM
	\item Listen to Arduino code structure walk through
\end{todolist}

\section{6th Grade Tech Class Six TODO:}
\begin{todolist}
	\item Make Github Issue to create blink program
	\item Create blank INO in Grithub Repository Working Directory
	\item Open Blank INO in Arduino.
	\item Add empty setup and loop functions.
	\item Commit with issue (but not push)
	\item Use 'pinMode()' function to set pin as output
	\item Use 'digitalWrite()' to turn LED on
	\item Use the IDE to offer suggestions for parameters
	\item Use 'delay()' to pause program
	\item Use 'digitalWrite()' to turn LED off
	\item Use 'delay()' to pause program
\end{todolist}
\newpage
\section{6th Grade Tech Class Seven TODO:}
\begin{todolist}
	\item Bring Chromebook to class
	\item Follow along discussion on Chromebook safety
	\item Install Ublock Origin
	\item Find the drive folder shared with me
	\item Make a folder in the shared folder
	\item Observe that all edits are tracked with names and time
	\item Create a slideshow in folder
	\item Login to Ubuntu and open the slideshow
	\item Add photo from README to slideshow
	\item Take 3 new photos of bulldog puppy
	\item Crop and Resize all 3 photos
	\item add all photos to git and to slideshow
\end{todolist}

\section{6th Grade Tech Class Eight TODO:}
\begin{todolist}
	\item Do 20 minutes of Type to Learn
	\item Go over all of the keywords in the Keywords sheet
	\item Take time to appreciate how much has been accomplished so far
	\item Add a slide to the build log
\end{todolist}
\newpage

\section{6th Grade Tech Class Nine TODO:}
\begin{todolist}
	\item Make Issue to add Servos to Arduino
	\item Install ESP32Servo library
	\item Open Sweep Example for ESP32Library and run it
	\item Include ESP32Servo.h
	\item Make Servo variables
	\item Make Integer variables for the Servo's stop value
	\item Attach servos to 33 and 32
	\item Write servos to the stop value
	\item Commit changes with Issue in message
	\item Open Chromebook and add a slide to the build log
\end{todolist}

\section{6th Grade Tech Class Ten TODO:}
\begin{todolist}
	\item Open Google Slides and add a new slide
	\item Screen-capture the steps of the previous class
	\item Use Krita to crop and resize
	\item Use Krita to add circles around important elements in the image
	\item Save images in Git local repository folder
	\item Add context to images in Slides 
\end{todolist}
\newpage
\section{6th Grade Tech Class Eleven TODO:}
\begin{todolist}
	\item Make Issue to add WiiNunchuck to Arduino
	\item Install WiiChuck library
	\item Open Example for WiiChuck and run it
	\item Open the Serial Monitor and observe print statements
	\item Add Wii Accessory variable to Arduino program. 
	\item Use 'map()' function to convert values to -1 to 1 range
	\item Calculate a motor speed based on joystick value
	\item Make your program move your bulldog puppy forward and backwards
	\item commit changes with Issue in themessage
	\item Open Chromebook and add slide to build log
\end{todolist}
\newpage
\section{6th Grade Tech Class Twelve TODO:}
\begin{todolist}
	\item Make Issue to add Turning to Bulldog Puppy control
	\item Change scale of forward/back input to "45,135" instead of "0,180"
	\item Upload and observe slower driving, better control
	\item Make 2 new map() calls using the right/left joystick (Index 0 instead of index 1)
	\item Scale the left/right map to be range -45,45
	\item Change Servo.write() calls to be 90 + turnVariable
	\item Upload and observe the turning
	\item Change Servo.write() to be speedVariable + turnVariable
	\item Upload and observe drive and turn
	\item Commit changes with Issue \# in commit message
	\item Open Build log and add slides for turning + driving
\end{todolist}
\newpage
\section{6th Grade Tech Class Thirteen TODO:}
\begin{todolist}
	\item Make Issue to add the BNO055 IMU
	\item Open Library manager and add EasyBNO055\_ESP and all dependent libraries  
	\item Open EsayBNO055\_ESP example EsayBNO055\_ESP\_test and upload it
	\item Observe print statements as you turn the bulldog puppy
	\item Make a variable for IMU target, set to 0 to begin with
	\item Compute the error between current heading and IMU target
	\item Print the error and current heading to Serial Monitor
	\item Make a float variable for P gain
	\item Make a variable for control signal that is error times P gain
	\item Use the Constrain function (Google Arduino Constrain Reference) for the control signal
	\item Change the joystick control to use the control signal instead of the joystick input
	\item Upload program and observe the bulldog puppy maintains heading when moved on a piece of paper
	\item Commit with the Issue \# in the message and push
	\item Open Build log and add slides Adding IMU control
\end{todolist}

\section{6th Grade Tech Class Fourteen TODO:}
\begin{todolist}
	\item Make Issue to Combine Joystick Control with IMU
	\item Change the map() method for the "Joystick side to side" to be a range -100 to 100, stored to a float orientation update variable
	\item Divide the orientation update variable by 100.0
	\item Add the orientation update variable to the orientation target variable 
	\item Add a serial print to observe the orientation target and orientation update
	\item Observe orientation target changes with the left/right of the joystick
	\item Add in the forward/backwards to the turn terms 
	\item Upload and observer straight driving and clean turning
	\item Commit with the Issue \# in the message and push
	\item Open Build log and add slides Adding IMU control with joystick control
\end{todolist}

\end{document}
